#done in overleaf

\documentclass{article}
\usepackage{amsmath}
\usepackage{geometry}
\geometry{margin=1in}

\title{Gravitational Waves from Inspiralling Compact Binaries: Equation Summary}
\author{AK}
\date{15 july}

\begin{document}

\maketitle

\section*{Key Equations and Variable Definitions}

Modelled inspiral of two compact objects (black holes) using mathematical equations. The orbital dynamics are governed by the following (coupled) differential equations:

\subsection*{Velocity}

\begin{equation}
\frac{dv}{dt} = -\frac{F(v)}{dE/dv}
\end{equation}

\begin{itemize}
  \item \( v(t) \): Orbital velocity parameter (related to orbital frequency)
  \item \( F(v) \): Gravitational-wave energy flux (power radiated)
  \item \( E(v) \): Orbital's binding energy
\end{itemize}

\subsection*{Orbital Phase}

\begin{equation}
\frac{d\phi}{dt} = \frac{v^3}{m}
\end{equation}

\begin{itemize}
  \item \( \phi(t) \): Orbital phase (radians)
  \item \( m = m_1 + m_2 \): Total mass of the binary system
\end{itemize}

\subsection*{Gravitational-Wave Polarizations}

\begin{align}
h_+(t) &= 4 \cdot \frac{\mu}{m} \cdot v^2(t) \cdot \cos(\phi(t)) \\
h_{\times}(t) &= 4 \cdot \frac{\mu}{m} \cdot v^2(t) \cdot \sin(\phi(t))
\end{align}

\begin{itemize}
  \item \( h_+(t), h_{\times}(t) \): Plus and cross polarizations of the gravitational wave
  \item \( \mu = \frac{m_1 m_2}{m} \): Reduced mass
\end{itemize}

\subsection*{Other}

\begin{equation}
E(v) = -\frac{1}{2} \mu v^2 \quad \Rightarrow \quad \frac{dE}{dv} = -\mu v
\end {equation}

\begin{equation}
F(v) = \frac{32}{5} \left( \frac{\mu}{m} \right)^2 v^{10}
\end {equation}

\begin{itemize}
  \item These are the leading-order post-Newtonian expressions for energy and flux.
\end{itemize}

\subsection *{Initial Conditions}

\begin{itemize}
  \item \( v(t=0) = v_0 = 0.3 \)
  \item \( \phi(t=0) = 0 \)
  \item \( m_1 = m_2 = 5 M_{\odot} \) (solar masses)
\end{itemize}

\end{document}